*Relazione tra densit\'a dei nucleoni e impulso di Fermi
$A=\int_0^{+\infty}g(k)n(k)4\pi k^2dk$, $n(k)$ \'e il numero medio di occupazione dei livelli di singola particella: per gas completamente degenere (T=0)    
 $n(k)=\theta (k_F-k)$. 
 
Quindi: $A=\int_0^{k_F}g(k)4\pi k^2dk=\nu \frac{V}{(2\pi)^3}4\pi \frac{k_F^3}{3}$, e la densit\'a dei nucleoni \'e:
 
 \begin{equation*}
 \rho =\frac{A}{V}=\nu \frac{k_F^3}{6\pi^2}
 \end{equation*}

Da $\rho=\frac{A}{V}=\frac{A}{\frac{4\pi}{3}r_0^3A}$ ottengo:

$K_F=(\frac{9}{8}\pi)^{\frac{1}{3}}\frac{1}{r_0} (\nu=4)$.

La densit\'a (centrale) $\rho=0,17 Nucleone/fm^3$ quindi $K_F=(\frac{9}{8}\pi)^{\frac{1}{3}}(\frac{\frac{3}{\rho_0}}{4\pi})^{-\frac{1}{3}}=1,36 fm^{-1}$.

*Relazione fra densit\'a ed energia di Fermi

Energia di Fermi = energia dello stato occupato pi\'u alto: $\epsilon_F=\frac{\hbar^2k_F^2}{2m}\approx 38,35 MeV$ (Usando il $k_F$ della sezione di sopra).

Sostituendo l'espressione per $k_F$ ho:

$\epsilon_F=\frac{\hbar^2}{2m}\rho^{\frac{2}{3}}(\frac{6\pi^2}{\nu})^{\frac{2}{3}}$.

*Energia cinetica per particella

Determino l'energia media per nucleone:

$\epsilon_{Kin}=\frac{1}{2m_N}\frac{\int_0^{k_F}k^4dk}{\int_0^{k_F}k^2dk}=\frac{3}{5}\frac{p_F^2}{2m_N}$, o pi\'u direttamente:

$\frac{E_{Kin}}{A}=\frac{3}{5}\epsilon_F \approx 24MeV$.

$\frac{E}{A}=- \frac{B}{A}=-a_V+\text{Trascuto gli altri termini}\approx-15 MeV$ ed essendo l'energia cinetica per particella $(\frac{E}{A})_{Cin}=24 MeV$ derivo che il potenziale di singola particella \'e $(\frac{E}{A})_{Pot}=-39MeV$.