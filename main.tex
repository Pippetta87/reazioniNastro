\makeatletter
\let\@starttocorig\@starttoc
\makeatother%%

\documentclass[10pt,xcolor={usenames},fleqn,mathserif,serif]{beamer}

%%%Usefull link
%tikz-equations:
%http://www.wekaleamstudios.co.uk/posts/creating-a-presentation-with-latex-beamer-equations-and-tikz/

%% colors
\definecolor{bittersweet}{rgb}{1.0, 0.44, 0.37}
\definecolor{brilliantlavender}{rgb}{0.96, 0.73, 1.0}
\definecolor{antiquefuchsia}{rgb}{0.57, 0.36, 0.51}
\definecolor{violetw}{rgb}{0.93, 0.51, 0.93}
\definecolor{Veronica}{rgb}{0.63, 0.36, 0.94}
\definecolor{atomictangerine}{rgb}{1.0, 0.6, 0.4}
\definecolor{darkgray}{rgb}{0.66, 0.66, 0.66}
\definecolor{brightcerulean}{rgb}{0.11, 0.67, 0.84}
\definecolor{cadmiumorange}{rgb}{0.93, 0.53, 0.18}
\definecolor{ochre}{rgb}{0.8, 0.47, 0.13}
\definecolor{midnightblue}{rgb}{0.1, 0.1, 0.44}
\definecolor{lemon}{rgb}{1.0, 0.97, 0.0}
\definecolor{grey}{rgb}{0.7, 0.75, 0.71}
\definecolor{amber}{rgb}{1.0, 0.75, 0.0}
\definecolor{almond}{rgb}{0.94, 0.87, 0.8}
\definecolor{bf}{RGB}{88, 86, 88}
\definecolor{bb}{RGB}{177, 177, 177}
\definecolor{keyword}{rgb}{0.25, 0.25, 0.28}

%%%%%%%%%%%%%%%%%%%%%%%%%%%%%%%%%%% importa pacchetti
\usepackage{usepkg}
%%%%%%%%%%%%%%%%%%%%%%%%%%%%%%%%%%% Funzioni generali
\usepackage{functions}
%http://tex.stackexchange.com/questions/246/when-should-i-use-input-vs-include
\newcommand{\setmuskip}[2]{#1=#2\relax} %%problem usinig mu with calc (req by mathtools) loaded
\usepackage{sources}
%\usepackage{length}
%%%%%%%%%%%%%%%%%%%%%%%%%%%%%%%%%%% Funzioni per questo file main
\usepackage{mathOp}

%%tikz sources

%%%%
\def\status{coazione}
\def\keeptrying{coazione}
\usepackage{LocalF}
%%%%%%%%%%%%%%%%%%%%%%%%%%%%%%%%%
\usepackage{beamersetup}
%% Titolo
\title{Interazioni fondamentali}

% Let's get started
\begin{document}


\begin{filecontents}{coulombdistant.tex}


\begin{figure}
    \centering
\begin{tikzpicture}
    \coordinate (c) at (0,0) ;
\draw[->] (3,0) node[above] {$\vec{k_i}= \frac{\vec{p_i}}{\hbar}$} -- (0,0) ;
\draw[->] (0,0)--(0.5,-0.5);
\draw (0.3,-0.2) node  {$\vec{x}'$};
\draw[->]  (0,0)--(-2,0.5);
\draw (-0.2,0.2) node[above] {$\vec{x}$};
\draw[->] (0.5,-0.5) --(-2,0.5) node[above] {$\Pelectron$};
\draw (-1.3,-0.2) node[above] {$\vec{x}-\vec{x'}$};
\draw[black,fill=gray,opacity=0.4, rounded corners=1mm] (c) \irregularcircle{1cm}{2mm};

\draw (0.5,-0.5) +(-2pt,-2pt) rectangle +(2pt,2pt) node[below] {$dQ$};

\node at (0,-5) {\parbox{0.7\textwidth}{
$dV=-e\frac{dQ}{|\vec{x}-\vec{x'}|} \Rightarrow -ze^2\int \frac{\rho_c(\vec{x'})}{|\vec{x}-\vec{x'}|}d^3x'=V(r)$\\
$r=|\vec{x}|\gg r'=|\vec{x'}|$: 
\lbt{|\vec{x}-\vec{x'}|\approx r-\hat{r}\cdot\vec{x'}}{\frac{1}{|\vec{x}-\vec{x'}|} \approx r}
}};
\node (caption) at (0,-8) {\parbox{0.8\textwidth}{\caption{Potenziale Coulombiano}\label{fig:couldistw}}};

\end{tikzpicture}
    
\end{figure}

\end{filecontents}%%contain tikz files as filecontents

\begin{wordonframe}{work in progress}

Choose the way!!

parte tools:
    \begin{itemize}
        \item \keyword{TODO: math elements}: funzioni speciali
        \item \keyword{TODO: QM} 
        \item \keyword{TODO: scattering, urti, decadimenti}
        \item \keyword{TODO: EM relativit\'a, particelle cariche e materia}: decadimenti gamma e processo inverso
    \end{itemize}

parte forze nucleari: 
    \begin{itemize}
    \item \keyword{TODO: fisica nucleare}:Interazione forte. deutone, modelli di nucleo, scattering NN. 
    \item decadimento beta
    \item modelli nucleari
    \item Interaione EM/nucleo
    \end{itemize}
    
Parte reazioni nucleari nelle stelle:
\begin{itemize}
    \item Catena PP. Bi-Ciclo CNO.
    \end{itemize}

\end{wordonframe}

\begin{frame}
  \titlepage
  \tableofcontents[onlyparts]

\end{frame}

% Section and subsections will appear in the presentation overview
% and table of contents.
%\frame{\tableofcontents[onlyparts]}
%\begin{frame}{Argomenti}
%  \tableofcontents[part=1,hideallsubsections%,pausesections
%  ]
%  % You might wish to add the option [pausesections]
%\end{frame}

\part{Intro. Viviani/Marcucci blowing}\linkdest{lezioni}
\section{RegLez}
\begin{frame}[allowframebreaks]{Registro 19}
\phantomsection\linkdest{rl18}
\cite{reg18}.
\listofkeywords

\begin{itemize}
\item 
\end{itemize}
\end{frame}

\begin{frame}[allowframebreaks]{Reg Lez 18}
\begin{itemize}  

    Mer 14/02/2018 11:00-13:00 (2:0 h) lezione: Introduzione al corso. Ripasso di alcuni aspetti di fisica nucleare pertinenti al corso: energia di legame e andamento in funzione del numero di massa A; carta dei nucleidi; massa nucleare in funzione del numero di protoni Z; parabola di massa e decadimenti beta. Decadimenti nucleari. Caratteristiche generali della forza nucleare. (Laura Elisa Marcucci)
    Gio 15/02/2018 09:00-11:00 (2:0 h) lezione: Il deutone: dati sperimentali, calcolo della funzione d'onda del deutone con una buca di potenziale centrale (calcolo analitico), e con un potenziale centrale generico (calcolo numerico). Metodi numerici per risolvere l'equazione di Schroedinger: algoritmo di Numerov. (Laura Elisa Marcucci)
    Gio 15/02/2018 14:00-16:00 (2:0 h) lezione: La funzione d'onda del deutone con potenziale non centrale: componente con L=0 e L=2 della funzione d'onda. Discussione dettagliata della parte dipendente dallo spin e dal momento angolare orbitale della funzione d'onda ({\cal Y}_{LSJJz}). Soluzione asintotica. Calcolo del momento di dipolo magnetico e probabilita` di onda D. (Laura Elisa Marcucci)
    Mer 21/02/2018 09:00-11:00 (2:0 h) lezione: Scattering nucleone-nucleone a basse energie (onda s): scattering da buca di potenziale, sfasamento, relazione tra il segno dello sfasamento e il tipo di potenziale (attrattivo o repulsivo). Relazione tra sezione d'urto e sfasamento, sezione d'urto di singoletto e di tripletto; definizione della lunghezza di scattering; valori per la lunghezza di scattering di singoletto e di tripletto e conseguenti informazioni sulla forza nucleare. Charge-independence della forza nucleare. Calcolo del range della forza nucleare a cui la forza diventa da attrattiva a repulsiva. Cenni al modello di forza dovuto allo scambio di un pione. (Laura Elisa Marcucci)
    Gio 22/02/2018 09:00-11:00 (2:0 h) lezione: Decadimento β: tipi di decadimento spettri degli elettroni emessi e l'ipotesi di emissione dei neutrini; energetica del decadimento β; il Q della reazione; il decadimento del neutrone e prima stima della massa del neutrino. (Michele Viviani)
    Gio 22/02/2018 14:00-16:00 (2:0 h) lezione: Decadimento β: dettagli sull'interazione, termine di Fermi e di Gamow-Teller; teoria di Fermi e l'interazione di contatto; elemento di matrice e l'approssimazione delle transizioni permesse; calcolo della probabilita' di emissione di un elettrone con una data energia, calcolo dello spazio delle fasi, correzioni Coulombiane (cenni); quantita' ft; plot di Kurie; decadimenti superpemessi e stima della costante d'accoppiamento di Fermi gF; decadimento del neutrone e stima del costante gGT; regole di selezione delle trasizioni di Fermi e Gamow-Teller; cenni sulle transizioni proibite. (Michele Viviani)
    Mer 28/02/2018 09:00-11:00 (2:0 h) lezione: Decadimento gamma: energetica del decadimento gamma; cenni della teoria classica di radiazione; interazione fotoni-nucleoni dalla sostituzione minimale; corrente; calcolo della probabilita' di decadimento dalla teoria delle perturbazioni al primo ordine; espansione in multipoli (cenni); multipoli elettrici e magnetici; regole di selezione per transizioni di tipo elettrico e magnetico. (Michele Viviani)
    Gio 01/03/2018 09:00-11:00 (2:0 h) non tenuta: Sospensione attivita' didattica causa neve (Michele Viviani)
    Gio 01/03/2018 14:00-16:00 (2:0 h) non tenuta: Sospensione attivita' didattica causa neve (Michele Viviani)
    Mer 07/03/2018 09:00-11:00 (2:0 h) lezione: Decadimento gamma: stime di Weisskopf della probabilita' di decadimento nei vari multipoli; esempio di decadimenti: caso del 60Co e 137Cs; stati isomerici. Modello a shell per la struttura nucleare; numeri magici di Z e N; ipotesi di riempimento dei livelli di particella singola; Hamiltoniana di modello a shell; autostati come determinanti di Slater. (Michele Viviani)
    Gio 08/03/2018 09:00-11:00 (2:0 h) lezione: Modello a shell: potenziale di particella singola di tipo oscillatore armonico; livelli e loro classificazione in notazione spettroscopica; potenziale di tipo Wood-Saxon e di tipo spin-orbita; autostati del momento angolare j=l+s; riproduzione dei numeri magici; predizione dello spin dello stato fondamentale e dei primi stati eccitati dei nuclei con A=dispari. (Michele Viviani)
    Gio 08/03/2018 14:00-16:00 (2:0 h) lezione: Calcolo dei momenti magnetici dei nuclei co A=dispari con il modello a shell; linee di Schmidt; discussione di alcuni casi particolari: nuclei 3H, 3He, 15N, 207Pb. Stati collettivi: caso del nucleo 130Sn; stati di due particelle in uno shell j; energia del primo stato eccitato 2+; interpretazione di questo stato come di un "fonone" (cenni); banda rotaziopnali per nuclei non sferici (cenni) (Michele Viviani)
    Mer 14/03/2018 09:00-11:00 (2:0 h) lezione: Osservazioni astronomiche alla base dei modelli astrofisici in cui sono fondamentali gli input di fisica nucleare: caratteristiche della Via Lattea, massa e luminosita`. Diagramma HR: discussione qualitativa. Legge di Hubble ed eta` dell'universo. Cenni alla teoria del Big Bang: descrizione delle varie tappe fino alla fase di freeze-out. Rapporto tra numero di neutroni e numero di protoni all'inizio della fase della Big Bang Nucleosynthesis (BBN). Stima della mass fraction per l'4He. Input fondamentali per la teoria della BBN. Network delle reazioni della Big Bang Nucleosynthesis: elenco delle 12 reazioni chiave che portano all'abbondanza primordiale per i nuclei di 4He, 2H, 3He e 7Li. (Laura Elisa Marcucci)
    Gio 15/03/2018 09:00-11:00 (2:0 h) lezione: Discussione delle 12 reazioni chiave della Big Bang Nucleosyntheis che portano all'abbondanza primordiale per i nuclei di 4He, 2H, 3He e 7Li. Discussione della mass fraction in funzione del tempo e in funzione della densita` barionica. Paragone con i dati sperimentali per le abbondanze primordiali. Il "puzzle" per il 7Li. Paragone con i dati di Planck 2015: ulteriori "tensioni" tra teoria ed esperimento. Potere predittivo della BBN: calcolo del numero di sapori dei neutrini. Studio dettagliato della reazione n+p --> d + gamma: sviluppo della cinematica della reazione. (Laura Elisa Marcucci)
    Gio 15/03/2018 14:00-16:00 (2:0 h) lezione: Studio dettagliato della reazione n+p --> d + gamma: sviluppo dell'elemento di matrice della corrente nucleare nell'ipotesi di cattura dallo stato di scattering np 1S0, includendo solo lo stato 3S1 del deutone e suo calcolo esplicito nel caso di buca di potenziale. Paragone con i dati sperimentali: pseudo-ortogonalita' e importanza dello stato 3D1. Introduzione agli operatori di corrente di scambio. (Laura Elisa Marcucci)
    Mer 21/03/2018 09:00-11:00 (2:0 h) lezione: Nucleosintesi stellare: la catena protone-protone. Presentazione e discussione sulle reazioni presenti nella catena. Flusso dei neutrini solari. Discussione degli esperimenti di R. Davis (Homestake), Gallex, Kamiokande e Super-Kamiokande, SNO e Borexino. Paragone tra teoria ed esperimento e soluzione del problema dei neutrini solari con l'introduzione dell'oscillazione dei neutrini. (Laura Elisa Marcucci)
    Gio 22/03/2018 09:00-11:00 (2:0 h) lezione: Calcolo del rate di reazione in assenza di risonanze. Dipendenza della sezione d'urto dall'energia e definizione del fattore astrofisico. Definizione del picco di Gamow e calcolo dell'energia del picco di Gamow per la reazione protone+protone. Definizione del parametro tau. Espansione in 1/tau nell'intergale del rate. Espressione finale in termini di S_eff e discussione del risultato. (Laura Elisa Marcucci)
    Gio 22/03/2018 14:00-16:00 (2:0 h) lezione: Rate di una reazione in presenza di una risonanza: definizione di risonanza e derivazione della formula di Breit-Wigner per la sezione d'urto in un approccio semi-classico. Definizione di risonanza nel caso di sistema unidimensionale con potenziale a buca attrattiva. Calcolo del rate e discussione della formula finale. Discussione dell'andamento del fattore astrofisico in presenza di risonanze larghe, strette, sopra e sotto soglia. (Laura Elisa Marcucci)
    Mer 11/04/2018 09:00-11:00 (2:0 h) lezione: Derivazione esplicita del fattore di Gamow: calcolo esplicito del coefficiente di trasmissione nel caso di scattering uni-dimensionale con potenziale attrattivo per x minore di x0, e barriera repulsiva costante tra x0 e xb. Derivazione del fattore di Gamow per potenziale Coulombiano. (Laura Elisa Marcucci)
    Gio 12/04/2018 09:00-11:00 (2:0 h) lezione: Screening elettronico nel plasma stellare: effetto sul rate delle reazioni nelle stelle; calcolo del potenziale di screening nel caso particolare di screening debole, raggio di Debey-Hueckel e formula finale per lo screening. Screening elettronico negli esperimenti in laboratorio: effetto sulla sezione d'urto misurata e risalita esponenziale a piccole energie. (Laura Elisa Marcucci)
    Gio 12/04/2018 14:00-16:00 (2:0 h) lezione: Studio della reazione di cattura pp; interazione responsabile; calcolo della probabilita' di transizione usando la regola d'oro di Fermi; elemento di matrice: contributo dall'operatore di Gamow-teller; calcolo dello spazio delle fasi ed espressione della sezione d'urto; analisi dimensionale (Michele Viviani)
    Mer 18/04/2018 09:00-11:00 (2:0 h) lezione: Studio della soluzione dell'equazione di Schroedinger per un sistema a due corpi interagenti tramite potenziale Coulombiano; la funzione ipergeomtrica confluente; definizione, e proprieta' asintotiche; l'ampiezza di scattering esatta per utro puramente Coulombiano; la sezione d'urto di Rutherford. (Michele Viviani)
    Gio 19/04/2018 09:00-11:00 (2:0 h) lezione: Decomposizione della funzione di scattering ottenuta nella lezione precedente in onde parziali; la funzione regolare di Coulomb F e lo sfasamento Coulombiano; soluzione irregolare G e loro proprieta' a piccoli r ed asintotiche. Soluzione dell'equazione di Schroedinger per due corpi in presenza di potenziale di Coulomb e potenziale nucleare a corto raggio. Applicazione alla cattura pp per un potenziale nucleare di tibo buca quadrata seguendo il calcolo di Bethe e Critchfield del 1938; calcolo della funzione di scattering pp in onda S e di quella del deutone. (Michele Viviani)
    Gio 19/04/2018 14:00-16:00 (2:0 h) lezione: Continuazione del calcolo della lezione precedente; calcolo dell'elemento di matrice e quindi della sezione d'urto e del fattore astrofisico della cattura pp; stima numerica dei risultati; stima del rate di reazione e dell'energia emessa dal sole; cenni sui calcoli recenti del fattore astrofisico della cattura pp. Analisi delle altre reazioni della catena pp: misure sperimentali e calcoli teorici piu' recenti (Michele Viviani)
    Gio 26/04/2018 09:00-11:00 (2:0 h) lezione: Metodi di calcolo della funzione d'onda per A>2; i vettori di Jacobi e la separazione del moto del centro di massa; problemi collegati all'espanzione in onde parziali e alla antisimmetria (cenni); il metodo variazionale ed il Green Function Monte Carlo per gli stati legati (cenni); metodo di calcolo degli stati del continuo; il metodo della matrice R; separazione del problma in una parte esterna asintotica e una interna; l'operatore di Block (Michele Viviani)
    Gio 26/04/2018 14:00-16:00 (2:0 h) lezione: Il metodi della matrice R (continua); definizione della matrice R e sua relazione con gli sfasamenti; formula per la matrice R in functione dell'energia e suo uso per fittare e poi estrapolare i dati sperimentali. Il metodo del bilancio dettagliato e della dissociazione Coulombiana; suo studio attraverso il metodo CDCC (cenni) (Michele Viviani)
    Mer 02/05/2018 09:00-11:00 (2:0 h) lezione: Tecniche sperimentali per l'astrofisica: esperimenti diretti; l'esperimento LUNA ai Laboratori Nazionali del Gran Sasso; riduzione della radiazione di background; fasi dell'esperimento fino a LUNA-MV; cenni su alcune misure effettuate nell'esperimento LUNA. Gli esperimenti con RMS (Recoil Mass Separator); principio di funzionamento; grafici energia-tempo di volo; gli esperimenti ERNA e CIRCE (cenni). (Michele Viviani)
    Gio 03/05/2018 09:00-11:00 (2:0 h) lezione: Tecniche sperimentali indirette: Trojan Horse Method (THM); principio di funzionamento; esempio di reazione studiato con il THM: la fusione d+d->3H+p. Il metodo delle Asymptotic Normalization Constants (ANC); definizione di ANC; la funzione di Whittaker; caso con lo spin; esempio di ANC per il nucleo di 4He; cenni di come determinare il fattore astrofisico a partire dall'ANC. (Michele Viviani)
    Gio 03/05/2018 14:00-16:00 (2:0 h) lezione: Il ciclo CNO: reazioni del ciclo principale e dei cicli secondari; stima della variazione nel tempo della popolazione dei nuclei X che effettuano una reazione tipo X(p,gamma)Y in un ambiente tipo il sole; esempio per il caso 2H(p,gamma)3He; discussione sulla reazione piu' lenta del ciclo CNO-1; discussione su i neutrini emessi nel ciclo CNO e loro rivelazione da apparati terrestri; dipendenza della temperatura dell'energia emessa nei cicli pp e CNO. (Michele Viviani)
    Mar 08/05/2018 09:00-11:00 (2:0 h) lezione: Le reazioni di fusione dell'He; la reazione triple-alpha per la formazione del 12C; formazione del 8Be e tempo medio tra una collisione ed un altra in una stella gigante rossa; l'ipotesi di Hoyle di uno stato eccitato del 12C in corrispondenza dell'energia del 8Be+4He; stima sulla posizione dello stato eccitato a partire dall'abbondanza del 12C nell'universo ed usando il rate di reazione calcolato dalla formula di Breit-Wigner; confronto con lo spettro degli stati eccitati del 12C. La reazione di fusione 12C+4He->16O+gamma; lo spettro degli stati eccitati del 16O; cattura risonante da stati soprasoglia e sottosoglia; cenni sulla situazione sperimentale ed importanza della nuova misura di LUNA-MV (Michele Viviani)
    Mer 09/05/2018 09:00-11:00 (2:0 h) lezione: La formazione di nuclei piu' pesanti del 56Fe; sequenze di cattura neutronica; confronto con la velocita' di cattura neutronica e tempo di decadimento beta degli isotopi formati; processi-r e processi-s; esempi di sequenze per il Fe e per lo stagno; breve discussione della sezione d'urto per cattura neutronica; legge "1/v"; legge n_A sigma_A= costante per processi-s; sequenza di reazioni risultanti in produzione di neutroni; discussione sulle stelle AGB come siti dei processi-s; ipotesi dei siti dei processi-r. Richiamo del ciclo pp e discussione sui rate delle varie reazioni di questo ciclo. (Michele Viviani)

\end{itemize}
\end{frame}

\part{Reazioni nucleari nelle stelle}\linkdest{mainaims}
\input{succo}


\part{Forze nucleari}\linkdest{nuclearforces}
\begin{frame}{this part toc}
\begin{itemize}
\item Strong force
\end{itemize}
\end{frame}
\input{nuclearforces}

\part{Modello standard}
\frame{\partpage}
\begin{frame}{this part toc}
\begin{itemize}
\item Catena PP
\end{itemize}
\end{frame}
\input{nrstars}


\part{List of sorted keywords}
\listofsortedkeywords

\end{document}